\documentclass[fleqn]{article}
\usepackage{amsmath}
\usepackage{amsfonts}
\usepackage{parskip}
\usepackage[left = 1.25in, right = 1.25in]{geometry}
\setlength{\parindent}{0pt}
\setlength{\mathindent}{20pt}

\begin{document}
	\title{\textbf{Combining PLONK with Plookup}}
	\author{
		Author 1 \\
		\texttt{Affiliation 1}\\
	\and
		Author 2\\
		\texttt{Affiliation 2}\\
	\and
		Author 3\\
		\texttt{Affiliation 3}\\
	}
	\date{May 2021}
	\maketitle
	
	\begin{abstract}
		In 2019, Gabizon, Williamson, and Ciobotaru introduced Plonk -- a fast and flexible zk-SNARK using an updatable and universal structured reference string. Plonk uses a \emph{grand product argument} to check permutations of wire values and exploits convenient interactions between multiplicative subgroups and Lagrange bases. The next year, Gabizon and Williamson used similar techniques to develop Plookup -- a zk-SNARK that can verify that each element in a list of queries can be found in a public lookup table. These two zk-SNARK constructions were presented seperately, but are similar enough to be combined into a single protocol. Here we demonstrate a combined Plonk and Plookup protocol, noting implementation details and justifying deviations from the original protocols required to combine them effectively.
	\end{abstract}
	\section{Introduction}
		%explain the basics of how Plonk and Plookup actualy work. The reader may know this information but it will also serve as n introduction to the terminologyn and goals as the authors see it
		Plonk uses constraints of the form:
			\[(q_L)_i\cdot a_i+(q_R)_i\cdot b_i+(q_O)_i\cdot c_i+(q_M)_i\cdot a_i b_i+(q_C)_i = 0\]
			where the vectors $\mathbf{q_L}, \mathbf{q_R}, \mathbf{q_O}, \mathbf{q_M},$ and $\mathbf{q_C}$ are \emph{selectors} which can scale their term in the constraint (or ``turn off'' that term by using a selector value of zero). The vectors $\mathbf{a}$,  $\mathbf{b}$, and $\mathbf{c}$ are \emph{wires} whose values at index $i$ need to satisfy the $i$th constraint. 
			
		A variable often needs to be used in multiple constraints, for instance the ``out'' value of constraint $i$ may need to be the ``left'' value in constraint $j$, i.e. $c_i = a_j$. These \emph{copy constraints} are enforced by a \emph{grand product argument} constructed using additional vectors representing permutations of variables.  
		
	\section{Terminology and Notation}
		%some of this may be moved into the Introduction later. 
		We use notation similar to GWC19, so that $\mathbb{F}$ is a field of prime order, $\mathbb{G}_1$, $\mathbb{G}_2$, and $\mathbb{G}_t$ are groups of order $r$, and $e$ is an efficiently computable non-degenerate pairing $e:\mathbb{G}_1\times\mathbb{G}_2\to\mathbb{G}_t$. There are generators for these groups, $g_1$, $g_2$, and $g_t$, such that $e(g_1, g_2) = g_t$.
	
		A $\emph{wire}$ is a vector of field elements providing the values of variables in each gate. Here four wires are used, labeled ``left", ``right", and ``out''. However, any number of wires can be used. In our implementation we use these same three wires to carry data for lookup queries so that our lookup table can represent a function of up to three inputs. A $\emph{selector}$ is also a vector of field elements which serve as coefficients to the variables in each gate. A value of $0$ at index $i$ in the multiplication selector means the multiplication term in gate $i$ is not being used (i.e. it is some other kind of gate).
		
	\section{Implementation Details}
		\subsection{Options for Hiding} % Add dummy values OR add multiple of Z_H
		In GWC19, hiding of the wire polynomials and the permutation polynomial is accomplished by adding a distinct random multiple of the target polynomial $Z_H$ to each polynomial that needs to be hidden. The random factors are either degree 1 or 2 and so the hidden polynomials have slightly higher degree than the unhidden polynomials.
		
		We found an alternative approach using randomized ``dummy" values added to each wire. This approach provides the same hiding as the method in GWC19, but is more useful in a context where padding the wires is usually necessary. The dummy values can also be used to pad the circuit.
		
		[[elaborate]]
		
		\subsection{Connecting Plookup to a Plonk Circuit} % Explain how the wire values are compressed into $f$ and rows that are not lookup rows need to be replaced with a dummy value
		For our needs, the Plookup and Plonk portions of the protocol needed to be able to refer to the same variables, so that the output variable of a Plonk gate could be an input to a Plookup gate and vice-versa. A simple way to achieve this is to use the same wires for both Plonk and Plookup gates. Then a \emph{selector} can be used to select only Plookup gates when necessary.
		
		\subsection{Padding for Sorting} % Must pad with a "middle" element to avoid constant polynomials for h1, h2
		In GW20, the vectors $f \in \mathbb{F}^{n-1}$ and $t \in \mathbb{F}^n$, containing the compressed lookup queries and the compressed lookup table rows respectively, are concatenated and sorted into a vector $s \in \mathbb{F}^{2n-1}$. Then $s$ is divided into two equal size parts, $h_1 \in \mathbb{F}^n$ and $h_2 \in \mathbb{F}^n$, so that the last (and largest) element of $h_1$ is the same as the first (and smallest) element of $h_2$. Then $f$, $t$, $h_1$, and $h_2$ are interpolated into polynomials and used to create the plookup permutation polynomial.
		
		Both $f$ and $t$ may need to be padded in order to get them to be the correct size. 
		
		However, if either $f$ or $t$ contains many repeats of the same value (say a certain query is repeated many times, or the lookup table is small and needs to be padded with many copies of a dummy row), and that value is very small or very large, it can happen that $h_1$ or $h_2$ consists entirely of the repeated element. When this vector is interpolated the resulting polynomial will be constant.
		
		[[Is this bad, apart from Dusk's code throwing an error? Will Kate commitments fail? Security broken?]]
		
		To prevent this, care must be taken to pad $f$ and $t$ with a field element that is neither the largest nor the smallest of either.
		
		\subsection{Constructing a Linearization Polynomial} % Show how linearization process works and can be extended to Plookup
		\subsection{The Query Polynomial} % mention how "short" and "long" versions of the query polynomial must be used in different places
		
		The vector $f$ contains the compressed queries from the wires. But $f$ is supposed to be length $n-1$ while each wire is supposed to be of length $n$. Some circuit row needs to be excluded when defining 
	\section{Combined Protocol}
		\subsection{Polynomials defining a circuit including Plookup}
			% wire polynomials
			% lookup table polynomials
			% lookup query polynomials
			% permutation polynomials
			\begin{itemize}
				\item The selectors $q_M(X), q_L(X), q_R(X), q_O(X), q_C(X)$ which define the arithmetization of a Plonk circuit
				\item The additional selector $q_K(X)$ which selects Plookup gates
				\item $S_{ID1}(X) = X, S_{ID2}(X) = k_1(X), S_{ID2}(X)=k_2X$: the identity permutation applied to $\mathbf{a}$, $\mathbf{b}$, $\mathbf{c}$, with constants $k_1, k_2\in \mathbb{F}$ chosen such that $H, k_1\cdot H, k_2\cdot H$ are distinct cosets in $mathbb{F*}$.
				\item $S_{\sigma_1}(X)$, $S_{\sigma_2}(X)$, $S_{\sigma_3}(X)$: The copy permutation applied to $\mathbf{a}$, $\mathbf{b}$ and $\mathbf{c}$.
				\item $n$, the number of gates for a given circuit (the sum total of arithmetic and lookup gates)
			\end{itemize}
		\subsection{Commitments}
			% KZG scheme
		\subsection{Preprocessing}
			% what can be preprocessed? what can't?
		\subsection{Prover algorithm}
			% mention relevant information from section "Implementation Details" where needed
			\subsubsection*{Round 1}
				\begin{enumerate}
					\item Generate random blinding scalars $(b_1,\ldots,b_9)\in \mathbb{F}_p$.
	
					\item Compute wire polynomials $a(X), b(X)$, and $c(X)$:         
					\[a(X) = (b_1X+b_2)Z_H(X)+\sum_{i=0}^{n-1} w_i\textbf{L}_i(X)\]					                
					\[b(X) = (b_3X+b_4)Z_H(X)+\sum_{i=0}^{n-1} w_{n+i}\textbf{L}_i(X)\]					                
					\[c(X) = (b_5X+b_6)Z_H(X)+\sum_{i=0}^{n-1} w_{2n+i}\textbf{L}_i(X)\]				                
					\item Compute $[a]_1$, $[b]_1$, and $[c]_1$ and add to transcript.
					                
					\item Generate table compression factor $\zeta$ from the transcript.
					                
					\item Compute compressed table $\left\{t_i\right\}_{i=0}^n$ as $t_i = x_{t_i}+\zeta y_{t_i}+\zeta^2z_{t_i}$ and sort.
					                
					\item Compute table polynomial $t(X) = \sum_{i=0}^{n-1}t_i \textbf{L}_i(X)$.
					                
					\item Compute compressed queries $\left\{f_i\right\}_{i=0}^n$ as $f_i = x_{f_i}+\zeta y_{f_i}+\zeta^2z_{f_i}$.
					                
					\item Compute query polynomial $f(X) = \sum_{i=0}^{n-1}f_i\textbf{L}_i(X)$.
					                
					\item Compute $[f]_1 = [f(x)]_1$.
				\end{enumerate}
					                
					Output $([a]_1, [b]_1, [c]_1, [f]_1)$.
            
			\subsubsection*{Round 2}
				\begin{enumerate}
					\item Compute permutation challenges $(\beta, \gamma, \delta, \varepsilon )\in \mathbb{F}_p$ from the transcript.
	    
					\item Compute permutation polynomial $z(X)$:
						\begin{align*} 
						z(X) &= (b_7X^2+b_8X+b_9)Z_H(X) + \textbf{L}_0(X) \\ 
						&+ \sum_{i=0}^{n-2}\left(\textbf{L}_{i+1}(X)\prod_{j=1}^{i}\frac{(w_j+\beta\omega^{j-1}+\gamma)(w_{n+j}+\beta k_1\omega^{j-1}+\gamma)(w_{2n+j}+\beta k_2\omega^{j-1}+\gamma)}{(w_j+\beta\sigma(j)+\gamma)(w_{n+j}+\beta\sigma(n+j)+\gamma)(w_{2n+j}+\beta\sigma(2n+j)+\gamma)}\right)
						\end{align*}
	                
					\item Compute $\left\{s_i\right\}_{i=0}^{2n-1}$, the sorted concatenation of $(f,t)$.
							                
					\item Compute the Plookup polynomial $p(X)$:	                
	               				\[p(X)= \textbf{L}_0(X) + \sum_{i=1}^{n-2} \left( \textbf{L}_i(X) \prod_{j=0}^{i-1}\frac{(\varepsilon+f_j)(1+\delta)(\varepsilon(1+\delta)+t_j+\delta t_{j+1})}{(\varepsilon(1+\delta)+{h_1}_j+\delta{h_1}_{j+1})(\varepsilon(1+\delta)+{h_2}_j+\delta{h_2}_{j+1})} \right)+ \textbf{L}_{n-1}(X)\]
	                
					\item Compute $[z]_1 = [z(x)]_1$, $[p]_1 = [p(x)]_1$, $[h_1]_1 = [h_1(x)]_1$, $[h_2]_1 = [h_2(x)]_1$.
				 \end{enumerate}

				Output $([z]_1, [p]_1, [h_1]_1, [h_2]_1)$.

			\subsubsection*{Round 3}
				\begin{enumerate}
					\item Compute challenges $\alpha_0, \alpha_1 \in \mathbb{F}_p$ from the transcript.
	
					\item Compute the quotient polynomial $Q(X)$:
						\[Q(X) = \frac{1}{Z_H(X)}\left(\;\begin{array}{ll}
						(a(X)b(X)q_M(X) + a(X)q_L(X) + b(X)q_R(X) + c(X)q_O(X) + PI(X) + q_C(X)) \\
						+(a(X)+\beta X+\gamma)   (b(X)+\beta k_1 X+\gamma)    (c(X)+\beta k_2X+\gamma)   z(X)  \alpha_0  \\
						-(a(X)+\beta S_{\sigma 1}(X)+\gamma)   (b(X)+\beta S_{\sigma 2}(X)+\gamma)  (c(X)+\beta S_{\sigma 3}(X)+\gamma)  z(X\omega)\alpha_0\\ 
						+(z(X)-1)\textbf{L}_0(X)\alpha_0^2 \\
						+ q_{Lookup}(X)(a(X)+\zeta b(X) + \zeta^2 c(X) - f(X))\alpha_1 \\
						+(p(X)-1)\textbf{L}_0(X)\alpha_1^2 \\
						+(X-\omega^{-1})p(X)(1+\delta)(\varepsilon+f(X))(\varepsilon(1+\delta)+t(X)+\delta t(X\omega))\alpha_1^3 \\
						-(X-\omega^{-1})p(X\omega)(\varepsilon(1+\delta)+h_1(X)+\delta h_1(X\omega))(\varepsilon(1+\delta)+h_2(X)+\delta h_2(X\omega))\alpha_1^3 \\
						+\textbf{L}_{n-1}(X)(h_1(X)-h_2(X\omega))\alpha_1^4 \\
						+\textbf{L}_{n-1}(X)(p(X)-1)\alpha_1^5
						\end{array}\right.\]
	                
					\item Split $Q(X)$ into degree $\leq n+2$ polynomials $Q_\text{lo}(X)$, $Q_\text{mid}(X)$, $Q_\text{hi}(X)$
				\end{enumerate}				

				Output $([Q_\text{lo}]_1, [Q_\text{mid}]_1, [Q_\text{hi}]_1)$.

			\subsubsection*{Round 4}
	    			\begin{enumerate}
					\item Compute evaluation challenge $\mathfrak{z}$ from the transcript.
					    
					\item Compute opening evaluations:
						\begin{align*}
						\bar{a} &= a(\mathfrak{z}), \bar{b} = b(\mathfrak{z}), \bar{c} = c(\mathfrak{z}), \bar{S}_{\sigma 1} = S_{\sigma 1}(\mathfrak{z}), \bar{S}_{\sigma 2} = S_{\sigma 2}(\mathfrak{z}), \\
						\bar{Q} &= Q(\mathfrak{z}), \bar{z}_\omega = z(\mathfrak{z}\omega), \bar{p}_\omega = p(\mathfrak{z}\omega),  \bar{f} = f(\mathfrak{z}), \bar{t} = t(\mathfrak{z}), \bar{t}_\omega = t(\mathfrak{z}\omega), \\
						\bar{h}_1 &= h_1(\mathfrak{z}), \bar{h}_{1_\omega} = {h}_1(\mathfrak{z}\omega), \bar{h}_{2_\omega} = {h}_2(\mathfrak{z}\omega)
						\end{align*}
					                
					\item Compute linearization polynomial $r(X)$:
						\begin{align*}
						r(X) &= \\
							&\phantom{+}\bar{a} \bar{b}\cdot q_M(X) + \bar{a}\cdot q_L(X) + \bar{b}\cdot q_R(X) + \bar{c}\cdot q_O(X) + q_C(X) \\
							&+((\bar{a} + \beta\mathfrak{z}+\gamma)(\bar{b} + \beta k_1 \mathfrak{z}+\gamma)(\bar{c} + \beta k_2\mathfrak{z}+\gamma)z(X))\alpha_0\\
							&-((\bar{a} + \beta\bar{s}_{\sigma 1}+\gamma)(\bar{b} + \beta\bar{s}_{\sigma 2}+\gamma)\beta\bar{z}_{\omega} \cdot S_{\sigma 3}(X)) \alpha_0 \\
							&+z(X)\textbf{L}_0(\mathfrak{z})\alpha_0^2 \\
							&-q_{Lookup}(X)\bar{f}\alpha_1 \\
							&+p(X)\textbf{L}_0(\mathfrak{z})\alpha_1^2 \\
							&+(\mathfrak{z}-\omega^{-1})p(X)(1+\delta)(\varepsilon+\bar{f})(\varepsilon(1+\delta)+\bar{t}+\delta \bar{t}_\omega)\alpha_1^3 \\
							&-(\mathfrak{z}-\omega^{-1})\bar{p}_\omega(\varepsilon(1+\delta)+\bar{h}_1+\delta \bar{h}_{1_\omega})h_2(X)\alpha_1^3\\
							&+\textbf{L}_{n-1}(\mathfrak{z})h_1(X)\alpha_1^4 \\
							&+\textbf{L}_{n-1}(\mathfrak{z})p(X)\alpha_1^5
						\end{align*}
		                
					\item Compute $\bar{r} = r(\mathfrak{z})$.
				\end{enumerate}

				Output $\bar{r}$.

			\subsubsection*{Round 5}
				\begin{enumerate}
					\item Compute opening challenge $v\in \mathbb{F}_p$ from the transcript.
					
					\item Compute opening proof polynomial $W_\mathfrak{z}(X)$:
						\[W_\mathfrak{z}(X) = \frac{1}{X-\mathfrak{z}}\left(\begin{array}{ll}(
						Q_\mathrm{lo}(X) + \mathfrak{z}^{n+2} Q_\mathrm{mid}(X) + \mathfrak{z}^{2n+4}Q_\mathrm{hi}(X) - \bar{Q})\\
						+ v(r(X)-\bar{r}) \\ 
						+ v^2(a(X)-\bar{a}) \\ 
						+ v^3(b(X)-\bar{b}) \\ 
						+ v^4(c(X)-\bar{c}) \\ 
						+ v^5(S_{\sigma 1}(X) - \bar{s}_{\sigma 1}) \\ 
						+ v^6(S_{\sigma 2}(X)  - \bar{s}_{\sigma 2}) \\
						+ v^7 (f(X) - \bar{f}) \\ 
						+v^8(h_1(X)-\bar{h}_1) \\ 
						\end{array}\right.\]
					    
					\item Compute opening proof polynomial $W_{\mathfrak{z}\omega}(X)$:				
						\[W_{\mathfrak{z}\omega}(X) =
						\frac{1}{X-\mathfrak{z}\omega}\left(\begin{array}{ll} z(X) - \bar{z}_\omega\\
						+ v'(h_1(X) - \bar{h}_{1\omega}) \\
						+ v'^2(h_2(X)-\bar{h}_{2\omega})\\
						+ v'^3(p(X)-\bar{p}_\omega)\\ 
						\end{array}\right.\]
					    
					\item Compute $[W_\mathfrak{z}]_1$ and $[W_{\mathfrak{z}\omega}]_1$.
				\end{enumerate}

				Output $([W_\mathfrak{z}]_1$, $[W_{\mathfrak{z}\omega}]_1)$.
				
				The full proof is:
					\[\pi_{\text{SNARK}} = \left(\;\begin{array}{ll} [a]_1, [b]_1, [c]_1, [f]_1, [z]_1, [p]_1, [h_1]_1, [h_2]_1, [Q_{\text{lo}}]_1, [Q_{\text{mid}}]_1, [Q_{\text{hi}}]_1, [W_\mathfrak{z}]_1, [W_{\mathfrak{z}\omega}]_1,\\
					\bar{a}, \bar{b}, \bar{c},\bar{s}_{\sigma 1}, \bar{s}_{\sigma 2},\bar{r}, \bar{z}_\omega, \bar{p}_\omega, \bar{h}_1, \bar{h}_{1_\omega}, \bar{h}_{2_\omega}, \bar{f}\end{array}\right)\]

		\subsection{Verifier Algorithm}
 
			\textbf{Verifier preprocessed input:} 

				$[q_M]_1 := q_M(x)\cdot[1]_1, [q_L]_1 := q_L(x)\cdot[1]_1,  [q_R]_1 := q_R(x)\cdot[1]_1, 
				[q_O]_1 := q_O(x)\cdot[1]_1, [q_\text{Lookup}]_1 := q_\text{Lookup}(x)\cdot[1]_1, [s_{\sigma 1}]_1 := s_{\sigma 1}(x)\cdot[1]_1, 
				[s_{\sigma 2}]_1 := s_{\sigma 2}(x)\cdot[1]_1, [s_{\sigma 3}]_1 := s_{\sigma 1}(3)\cdot[1]_1, [t]_1 := t(x)\cdot[1]_1, x\cdot[1]_2$

             		\begin{enumerate}
				\item Validate $([a]_1, [b]_1, [c]_1, [z]_1, [p]_1, [h_1]_1,[h_2]_1, [Q_{\text{lo}}]_1, [Q_{\text{mid}}]_1, [Q_{\text{hi}}]_1, [W_\mathfrak{z}]_1, [W_{\mathfrak{z}\omega}]_1)\in \mathbb{G}^{11}_1$.
				            
				\item Validate $(\bar{a}, \bar{b}, \bar{c}, \bar{z}, \bar{s}_{\sigma 1}, \bar{s}_{\sigma 2}, \bar{r}, \bar{z}_\omega, \bar{p}_\omega, \bar{h}_1, \bar{h}_{1_\omega}, \bar{h}_{2_\omega}, \bar{f}) \in \mathbb{F}_p^{13}$.
				            
				\item Validate $(w_i)_{i\in[l]}\in\mathbb{F}^l_p$.
				            
				\item Compute challenges $\beta, \gamma, \delta, \varepsilon, \alpha_0, \alpha_1, \mathfrak{z}, v, v', u \in \mathbb{F}_p$.
				            
				\item Compute zero polynomial evaluation $Z_H(\mathfrak{z}) = \mathfrak{z}^n-1$.
				            
				\item Compute Lagrange polynomial evaluations $\textbf{L}_0(\mathfrak{z}) = \frac{\mathfrak{z}^n-1}{n(\mathfrak{z}-1)}$ and $\textbf{L}_{n-1}(\mathfrak{z}) = \frac{\mathfrak{z}^n-1}{n(\mathfrak{z}\omega-1)}$.
				            
				\item Compute public input polynomial evaluation $PI(\mathfrak{z})=\sum_{i\in l}w_i\textbf{L}_i(\mathfrak{z})$, public table polynomial evaluations $\bar{t} = t(\mathfrak{z})$ and $\bar{t}_\omega = t(\mathfrak{z\omega})$, and public selector polynomial evaluation $\bar{q}_\text{Lookup} = q_\text{Lookup}(\mathfrak{z})$.
				            
				\item Compute the quotient polynomial evaluation:
					\[\bar{Q} =\frac{1}{Z_H(\mathfrak{z})}\left(\begin{array}{l}\bar{r} + PI(\mathfrak{z}) - ((\bar{a}+\beta\bar{s}_{\sigma 1}+\gamma)(\bar{b}+\beta\bar{s}_{\sigma 2}+\gamma)(\bar{c}+\gamma)\bar{z}_\omega)\alpha_0 \\ -\textbf{L}_0(\mathfrak{z})\alpha_0^2 + \left(\bar{q}_\text{Lookup}\right)(\bar{a}+\zeta \bar{b} + \zeta^2 \bar{c})\alpha_1 -\textbf{L}_0(\mathfrak{z})\alpha_1^2 \\ -(\mathfrak{z}-\omega^{-1})\bar{p}_\omega(\varepsilon(1+\delta)+\bar{h}_1+\delta \bar{h}_{1_\omega})(\varepsilon(1+\delta)+\delta \bar{h}_{2_\omega})\alpha_1^3 \\ -\textbf{L}_{n-1}(\mathfrak{z})\bar{h}_{2_\omega}\alpha_1^4-\textbf{L}_{n-1}(\mathfrak{z})\alpha_1^5 \end{array} \right.\]
	            
				\item Compute the first part of the batched polynomial commitment \\
					$\left[D\right]_1 := v\cdot[r]_1+u\cdot\left([z]_1 + v'[h_1]_1+v'^2[h_2]_1+v'^3[p]_1\right)$:
					\begin{align*}
					\left[D\right]_1 &= \bar{a}\bar{b}v\cdot[q_M]_1 + \bar{a}v\cdot[q_L]_1 + \bar{b}v\cdot[q_R]_1 + \bar{c}v\cdot[q_O]_1 + v\cdot[q_C]_1 -\bar{f}v\alpha_1\cdot[q_{\text{Lookup}}]_1 \\ 
					&+\left((\bar{a}+\beta\mathfrak{z}+\gamma)(\bar{b}+\beta k_1\mathfrak{z}+\gamma)(\bar{c}+\beta k_2 \mathfrak{z}+\gamma)\alpha v + \textbf{L}_0(\mathfrak{z})\alpha^2 v + u \right)\cdot[z]_1 \\ 
					&-(\bar{a}+\beta\bar{s}_{\sigma 1} + \gamma)(\bar{b}+\beta\bar{s}_{\sigma 2} + \gamma)\alpha v \beta \bar{z}_\omega[s_{\sigma 3}]_1 \\ 
					&+\left(\textbf{L}_{n-1}(\mathfrak{z})\alpha_1^4v+uv'\right)\cdot[h_1]_1 \\ 
					&-\left((\mathfrak{z}-\omega^{-1})\bar{p}_\omega(\varepsilon(1+\delta)+\bar{h}_1+\delta\bar{h}_{1_\omega})\alpha_1^3v + uv'^2\right)\cdot[h_2]_1 \\
					&+\left((\mathfrak{z}-\omega^{-1})(1+\delta)(\varepsilon+\bar{f})(\varepsilon(1+\delta)+\bar{t}+\delta\bar{t}_\omega)\alpha_1^3v + \textbf{L}_0(\mathfrak{z})\alpha_1^2v + \textbf{L}_{n-1}(\mathfrak{z})\alpha_1^5v +uv'^3\right)\cdot[p]_1
					\end{align*}
				
				\item Compute full batched polynomial commitment $[F]_1$:
					\begin{align*}	            
					[F]_1:&= [Q_\text{lo}]_1+\mathfrak{z}^{n+2}\cdot[Q_\text{mid}]_1+\mathfrak{z}^{2n+4}\cdot[Q_\text{hi}]_1 \\ 
					&+[D]_1+v^2\cdot[a]_1+v^3\cdot[b]_1+v^4\cdot[c]_1+v^5\cdot[s_{\sigma 1}]_1+v^6\cdot[s_{\sigma 2}]_1 \\
					&+v^7\cdot[f]_1+v^8\cdot[h_1]_1
					\end{align*}
					            
				\item Compute group-encoded batch evaluation $[E]_1$ :    
					\[[E]_1 := \left(\begin{array}{ll}\bar{Q} +v\bar{r} + v^2\bar{a} + v^3\bar{b} + v^4\bar{c} +v^5\bar{s}_{\sigma 1} +v^6\bar{s}_{\sigma 2} +v^7\bar{f}+v^8\bar{h}_1 \\ 
					+u(\bar{z}_\omega+v'\bar{h}_{1_\omega} + v'^2\bar{h}_{2_\omega}+v'^3\bar{p}_\omega)\end{array}\right)\cdot[1]_1\]
					            
				\item Batch validate all evaluations:
					\[e({W_\mathfrak{z}]_1}+u\cdot[W_{\mathfrak{z}\omega}]_1, [x]_2) \stackrel{?}{=} e(\mathfrak{z}\cdot [W_\mathfrak{z}]_1+u\mathfrak{z}\omega[W_{\mathfrak{z}\omega}]_1 + [F]_1 - [E]_1, [1]_2)\]
			\end{enumerate}
\end{document}

